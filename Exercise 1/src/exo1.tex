\section*{Exercise 1} % (fold)
\label{sec:section_name}

\subsection*{a)} % (fold)
\label{sub:a}

We know that the following equation holds

\begin{equation} \label{eq:length_dist}
	\sum_{N=1}^{\infty} p(N) = 1
\end{equation}

Knowing that we have a bigram language model, we can derive: 

\begin{align*}
	\sum_{N=1}^{\infty} p(N) \sum_{w_1^{N}} p(w_1^{N}) &= \sum_{N=1}^{\infty} p(N) \sum_{w_1^{N}} \prod_{i = 1}^{N} p(w_i|w_{i-1}) \\ 
	\text{(\textit{sum over all words in vocab} }	&= \sum_{N=1}^{\infty} p(N) \prod_{i = 1}^{N} \sum_{w_i}  p(w_i|w_{i-1}) \\ 
	&= \sum_{N=1}^{\infty} p(N) \prod_{i = 1}^{N} 1 \\ 
	&= \sum_{N=1}^{\infty} p(N) \\ 
	\text{(\textit{using equation \ref{eq:length_dist})} }	&= 1 
\end{align*}

\subsection*{b)} % (fold)
\label{sub:b}

\subsubsection*{i)}%
\label{ssub:i_}

The sentence end token on language model allows for language model 
normalization on word level independent of the length of the sentence. 
Without the sentence end token the following must always be true: 
\[
	\prod_{i=1}^{N} w_i = (\prod_{i=1}^{N-1} w_i) w_N \le \prod_{i=1}^{N-1} w_i
\]

The above equation logically holds, but having a corpus with a finite 
amount of sentences, it might very well be the case, that a sentence containing 
a subsentence is more likely to occur as a full sentence than the subsentence 
as a full sentence. 

Therefore, the sentence end model makes it possible to include sentence ends 
in the model and in doing so also ensures language model normalization on word 
level 
\[
	\sum_{w \in \mathbb{V} \cup \{\$\}} p(w|v) = 1, \forall v \in \mathbb{V}
\]

\subsubsection*{ii)}%
\label{ssub:ii_}


\begin{align*}
	\sum_{N=1}^{\infty}  \sum_{w_{1}^{N}: w_N = \$, w_n \in \mathbb{V}, \forall n = 1,...,N-1} p(w_1^{N}) &= \sum_{N=1}^{\infty}  \sum_{w_{1}^{N}: w_N = \$, w_n \in \mathbb{V}, \forall n = 1,...,N-1} \prod_{i = 1}^{N} p(w_i|w_{i-1})  \\
	 &= \sum_{N=1}^{\infty}  \prod_{i = 1}^{N-1} (\sum_{w_i} p(w_i|w_{i-1})) p(\$|w_{i-1}))  \\
	 &= \sum_{N=1}^{\infty}  \prod_{i = 1}^{N-1} (\sum_{w_i} p(w_i|w_{i-1})) p(\$)  \\
	\text{(\textit{using assumption}) }  &= \sum_{N=1}^{\infty}  \prod_{i = 1}^{N-1} (1-p(\$)) p(\$)  \\
	 &= \sum_{N=1}^{\infty}  (1-p(\$))^{N-1} p(\$)  \\
	\text{(\textit{using geometric series}) }	 &= \frac{1}{1 - (1 -p(\$))} p(\$)  \\
	 &= 1
\end{align*}

\subsubsection*{iii)}%
\label{ssub:iii_}

We can imply from ii) that 
\[
	p(N) = \sum_{w_{1}^{N}: w_N = \$, w_n \in \mathbb{V}, \forall n = 1,...,N-1} \prod_{i = 1}^{N} p(w_i|w_{i-1}) = (1-p(\$))^{N-1} p(\$) 
\]

